\documentclass{article}

%
% Import packages
% ===============

\usepackage[T1]{fontenc}
\usepackage{bytefield}
\usepackage{multicol}
\usepackage{geometry}
\usepackage{lipsum}
\usepackage{float}
\usepackage{parskip}

%
% Change default font
% ===================
\renewcommand{\familydefault}{\sfdefault}

\geometry{
	a4paper,
	total={170mm,257mm},
	left=20mm,
	top=20mm,
}


\title{MOD-I2C-GPIO}
\author{Olimex Ltd.}

\begin{document}
	\maketitle
	\tableofcontents
	\pagebreak
	
	\section{Description}
	\begin{multicols}{2}
		MOD-I2C-GPIO implements simple GPIO expander. The project is based on PIC16F18324.
		The device has the following capabilities:

	\end{multicols}
	

	\section{Memory map}
	\begin{multicols}{2}
	The device has memory map as shown on Figure \ref{fig:mem_map}.
	\end{multicols}
	\begin{figure}[H]
		\centering
		\begin{bytefield}[
			bitheight=2\baselineskip]{8}
			\begin{rightwordgroup}{Read-only}
				\bitbox[]{4}{\texttt{0x00}} & \bitbox{32}{Device ID} \\
				\bitbox[]{4}{\texttt{0x01}} & \bitbox{32}{Firmware version}
			\end{rightwordgroup} \\
				\bitbox[]{4}{\texttt{0x02}} & \bitbox{32}{Data direction} \\
			\begin{rightwordgroup}{Read-only}
				\bitbox[]{4}{\texttt{0x03}} & \bitbox{32}{Input data}
			\end{rightwordgroup} \\
				\bitbox[]{4}{\texttt{0x04}} & \bitbox{32}{Output data} \\
				\bitbox[]{4}{\texttt{0x05}} & \bitbox{32}{Pull-up enable} \\
				\bitbox[]{4}{\texttt{0x06}} & \bitbox{32}{Output mode} \\
				\bitbox[]{4}{\texttt{0x07}} & \bitbox{32}{Input buffer mode} \\
				\bitbox[]{4}{\texttt{0x08}} & \bitbox{32}{Input slew-rate control} \\
				\bitbox[]{4}{\texttt{0x09}} & \bitbox{32}{Interrupt enable} \\
				\bitbox[]{4}{\texttt{0x0A}} & \bitbox{32}{Interrupt sense high byte} \\
				\bitbox[]{4}{\texttt{0x0B}} & \bitbox{32}{Interrupt sense low byte} \\
			\begin{rightwordgroup}{Read-only}
				\bitbox[]{4}{\texttt{0x0C}} & \bitbox{32}{Interrupt status}
			\end{rightwordgroup} 
				
		\end{bytefield}
		\caption{Memory layout}
		\label{fig:mem_map}
	\end{figure}
	
	\section{Device registers}
	sdasd
	
	\subsection{Device ID}
	
		This register holds the unique device identification. It can be used for detection. The register is read-only, so it cannot change.
		
		\begin{tabular}{ l c }
			Address: & 0x00 \\
			Default: & 0x43 \\
		\end{tabular}
	
		\begin{figure}[H]
			\centering
			\begin{bytefield}[
				bitwidth=0.1\linewidth]{8}
				\bitheader[endianness=big, bitformatting={\small\bfseries}]{0-7} \\
				\bitbox{8}{ID}
			\end{bytefield}
			\caption{Device ID register}
			\label{reg:device_id}
		\end{figure}
	
	\subsection{Firmware version}
	
		Each new firmware release has its own revision. It's stored to this read-only register.
		The first release is 0x01, the second - 0x02, etc.
		
		\begin{tabular}{ l c }
			Address: & 0x01 \\
			Default: & - \\
		\end{tabular}
		
		\begin{figure}[H]
			\centering
			\begin{bytefield}[
				bitwidth=0.1\linewidth]{8}
				\bitheader[endianness=big, bitformatting={\small\bfseries}]{0-7} \\
				\bitbox{8}{FW}
			\end{bytefield}
			\caption{Firmware revision register}
			\label{reg:fw_rev}
		\end{figure}
	
	\subsection{Data direction}
	
		Each GPIO can be input or output. Setting a bit will make the corresponding pin input.
		Clearing it - output. 
	
		\begin{tabular}{ l c }
			Address: & 0x02 \\
			Default: & 0x00 \\
		\end{tabular}
		
		\begin{figure}[H]
			\centering
			\begin{bytefield}[
				bitwidth=0.1\linewidth]{8}
				\bitheader[endianness=big, bitformatting={\small\bfseries}]{0-7} \\
				\bitboxes{1}{{DIR7} {DIR6} {DIR5} {DIR4} {DIR3} {DIR2} {DIR1} {DIR0}}
			\end{bytefield}
			\caption{Data direction register}
			\label{reg:data_dir}
		\end{figure}
		
		\begin{itemize}
			\item \textbf{DIR7}: GPIO7 data direction control bit
			\begin{itemize}
				\item 1: Setup GPIO7 as input
				\item 0: Setup GPIO7 as output
			\end{itemize}
			\item \textbf{DIR6}: GPIO6 data direction control bit
			\begin{itemize}
				\item 1: Setup GPIO6 as input
				\item 0: Setup GPIO6 as output
			\end{itemize}
			\item \textbf{DIR5}: GPIO5 data direction control bit
			\begin{itemize}
				\item 1: Setup GPIO5 as input
				\item 0: Setup GPIO5 as output
			\end{itemize}
			\item \textbf{DIR4}: GPIO4 data direction control bit
			\begin{itemize}
				\item 1: Setup GPIO4 as input
				\item 0: Setup GPIO4 as output
			\end{itemize}
			\item \textbf{DIR3}: GPIO3 data direction control bit
			\begin{itemize}
				\item 1: Setup GPIO3 as input
				\item 0: Setup GPIO3 as output
			\end{itemize}
			\item \textbf{DIR2}: GPIO2 data direction control bit
			\begin{itemize}
				\item 1: Setup GPIO2 as input
				\item 0: Setup GPIO2 as output
			\end{itemize}
			\item \textbf{DIR1}: GPIO1 data direction control bit
			\begin{itemize}
				\item 1: Setup GPIO1 as input
				\item 0: Setup GPIO1 as output
			\end{itemize}
			\item \textbf{DIR0}: GPIO0 data direction control bit
			\begin{itemize}
				\item 1: Setup GPIO0 as input
				\item 0: Setup GPIO0 as output
			\end{itemize}
			
		\end{itemize}
		
	\subsection{Input data}
		
		This register holds input levels.
			
		\begin{tabular}{ l c }
			Address: & 0x03 \\
			Default: & - - - - \\
		\end{tabular}
			
		\begin{figure}[H]
			\centering
			\begin{bytefield}[
				bitwidth=0.1\linewidth]{8}
				\bitheader[endianness=big, bitformatting={\small\bfseries}]{0-7} \\
				\bitboxes{1}{{IN7} {IN6} {IN5} {IN4} {IN3} {IN2} {IN1} {IN0}}
			\end{bytefield}
			\caption{Input data register}
			\label{reg:input_value}
		\end{figure}
		
		\begin{itemize}
			\item \textbf{IN7}: GPIO7 input value
			\begin{itemize}
				\item 0: Input level on GPIO7 is low
				\item 1: Input level on GPIO7 is high
			\end{itemize}
			\item \textbf{IN6}: GPIO6 input value
			\begin{itemize}
				\item 0: Input level on GPIO6 is low
				\item 1: Input level on GPIO6 is high
			\end{itemize}
			\item \textbf{IN5}: GPIO5 input value
			\begin{itemize}
				\item 0: Input level on GPIO5 is low
				\item 1: Input level on GPIO5 is high
			\end{itemize}
			\item \textbf{IN4}: GPIO4 input value
			\begin{itemize}
				\item 0: Input level on GPIO4 is low
				\item 1: Input level on GPIO4 is high
			\end{itemize}
			\item \textbf{IN3}: GPIO3 input value
			\begin{itemize}
				\item 0: Input level on GPIO3 is low
				\item 1: Input level on GPIO3 is high
			\end{itemize}
			\item \textbf{IN2}: GPIO2 input value
			\begin{itemize}
				\item 0: Input level on GPIO2 is low
				\item 1: Input level on GPIO2 is high
			\end{itemize}
			\item \textbf{IN1}: GPIO1 input value
			\begin{itemize}
				\item 0: Input level on GPIO1 is low
				\item 1: Input level on GPIO1 is high
			\end{itemize}
			\item \textbf{IN0}: GPIO0 input value
			\begin{itemize}
				\item 0: Input level on GPIO0 is low
				\item 1: Input level on GPIO0 is high
			\end{itemize}
		\end{itemize}
		
		
	\subsection{Output data}
		
		This register sets output GPIO level.
		
		\begin{tabular}{ l c }
			Address: & 0x04 \\
			Default: & 0x00 \\
		\end{tabular}
		
		\begin{figure}[H]
			\centering
			\begin{bytefield}[
				bitwidth=0.1\linewidth]{8}
				\bitheader[endianness=big, bitformatting={\small\bfseries}]{0-7} \\
				\bitboxes{1}{{OUT7} {OUT6} {OUT5} {OUT4} {OUT3} {OUT2} {OUT1} {OUT0}}
			\end{bytefield}
			\caption{Output data register}
			\label{reg:output_value}
		\end{figure}
		
		\begin{itemize}
			\item \textbf{OUT7}: GPIO7 output value
			\begin{itemize}
				\item 0: Drive GPIO low
				\item 1: Drive GPIO high
			\end{itemize}
			\item \textbf{OUT6}: GPIO6 output value
			\begin{itemize}
				\item 0: Drive GPIO low
				\item 1: Drive GPIO high
			\end{itemize}
			\item \textbf{OUT5}: GPIO5 output value
			\begin{itemize}
				\item 0: Drive GPIO low
				\item 1: Drive GPIO high
			\end{itemize}
			\item \textbf{OUT4}: GPIO4 output value
			\begin{itemize}
				\item 0: Drive GPIO low
				\item 1: Drive GPIO high
			\end{itemize}
			\item \textbf{OUT3}: GPIO3 output value
			\begin{itemize}
				\item 0: Drive GPIO low
				\item 1: Drive GPIO high
			\end{itemize}
			\item \textbf{OUT2}: GPIO2 output value
			\begin{itemize}
				\item 0: Drive GPIO low
				\item 1: Drive GPIO high
			\end{itemize}
			\item \textbf{OUT1}: GPIO1 output value
			\begin{itemize}
				\item 0: Drive GPIO low
				\item 1: Drive GPIO high
			\end{itemize}
			\item \textbf{OUT0}: GPIO0 output value
			\begin{itemize}
				\item 0: Drive GPIO low
				\item 1: Drive GPIO high
			\end{itemize}
		\end{itemize}
			

	\subsection{Pull-up enable}
	
	All GPIOs has internal weak pull-up resistors. They are enabled by default, to minimize
	noise and power consumption. The can be disabled by either writing 0 to the corresponding
	bit or making the direction output.
	\par
	Special case is when GPIO is configure as open-drain. If the bit for given GPIO is set, then
	the pull-up is enabled is DAT bit is set. On DAT clear, the pull-up becomes inactive. 
	
	\begin{tabular}{ l c }
		Address: & 0x04 \\
		Default: & 0xFF \\
	\end{tabular}
	
	\begin{figure}[H]
		\centering
		\begin{bytefield}[
			bitwidth=0.1\linewidth]{8}
			\bitheader[endianness=big, bitformatting={\small\bfseries}]{0-7} \\
			\bitboxes{1}{{PU7} {PU6} {PU5} {PU4} {PU3} {PU2} {PU1} {PU0}}
		\end{bytefield}
		\caption{Pull-up control register}
		\label{reg:pullup}
	\end{figure}
	
	\begin{itemize}
		\item \textbf{PU7}: GPIO7 data direction control bit
		\begin{itemize}
			\item 0: Disables GPIO7 internal pull-up resistor
			\item 1: Enables GPIO7 internal pull-up resistor
		\end{itemize}
		\item \textbf{PU6}: GPIO6 data direction control bit
		\begin{itemize}
			\item 0: Disables GPIO6 internal pull-up resistor
			\item 1: Enables GPIO6 internal pull-up resistor
		\end{itemize}
		\item \textbf{PU5}: GPIO5 data direction control bit
		\begin{itemize}
			\item 0: Disables GPIO5 internal pull-up resistor
			\item 1: Enables GPIO5 internal pull-up resistor
		\end{itemize}
		\item \textbf{PU4}: GPIO4 data direction control bit
		\begin{itemize}
			\item 0: Disables GPIO4 internal pull-up resistor
			\item 1: Enables GPIO4 internal pull-up resistor
		\end{itemize}
		\item \textbf{PU3}: GPIO3 data direction control bit
		\begin{itemize}
			\item 0: Disables GPIO3 internal pull-up resistor
			\item 1: Enables GPIO3 internal pull-up resistor
		\end{itemize}
		\item \textbf{PU2}: GPIO2 data direction control bit
		\begin{itemize}
			\item 0: Disables GPIO2 internal pull-up resistor
			\item 1: Enables GPIO2 internal pull-up resistor
		\end{itemize}
		\item \textbf{PU1}: GPIO1 data direction control bit
		\begin{itemize}
			\item 0: Disables GPIO1 internal pull-up resistor
			\item 1: Enables GPIO1 internal pull-up resistor
		\end{itemize}
		\item \textbf{PU0}: GPIO0 data direction control bit
		\begin{itemize}
			\item 0: Disables GPIO0 internal pull-up resistor
			\item 1: Enables GPIO0 internal pull-up resistor
		\end{itemize}
		
	\end{itemize}

	
	

	
	
\end{document}